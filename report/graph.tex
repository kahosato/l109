In this section, I specify a weighted undirected graph $\mathcal{G}$ constructed from the Foursquare dataset and justify the design choices.

Recall that the purpose of this graph is to extract the notion of neighbourhood which is a set of venues that people would ``associate'' strongly with each other. I base the model for this association between two places on the transitions contained in the Foursquare dataset. The assumption is that, if one visits two places one after the other, there must be some relation between two places.

Concretely, a node in $\mathcal{G}$ represents a venue and an edge exists between two nodes if and only if the dataset contains a transition from one to the other. This concept of association should be invariant of the order of the visit. Therefore I disregard the direction of the transition and $\mathcal{G}$ is an undirected graph.

The weight of the edge then should represent the strength of association between two venues. In this report, we assume that one associate two locations more strongly if they are geographically close to each other. Based on this assumption, one may define the weight of an edge to be an inversely proportionate to a positive increasing function of distance. As the weight takes the multiple transitions between two venues into account, $\mathcal{G}$ has at most one edge between two nodes. $\mathcal{G}$ contains 86092 nodes and 1075409 edges as a result.

Some works attempt to relate spatial distance with non-spacial concepts, such as friendship. Letting $d$ to be the distance between two indivisuals, \cite{backstrom2010find} claims that the probability of them having a social connection is proportionate to $d^{-1}$. Others argue that it should be proportionate to $d^{-2}$~\citep{lambiotte2008geographical}. Though the context is different, here we assume that the strength of the association between two places decays in proportionate to the inverse of the distance.  For simplification, I define the distance between two venues to be the distance as the crow flies. One may elaborate this by, for instance, using how long it takes from one location to the other according to Google Maps.

It is also reasonable to assume that if two venues are associated strongly, more transitions should be made between them. In this report we assume that the strength of the association between two venues is in proportionate to how many transitions are recorded.

Therefore I define the weight of the edge between node $i$, $j$ to be:
\begin{align*}
w_{i, j} = \frac{transition\_count(i, j)}{distance(i, j)}
\end{align*}
where $transition\_count(i, j)$ returns the number of edges between node $i$, $j$, and $distance(i, j)$ returns the distance between venues represented by node $i$, $j$.