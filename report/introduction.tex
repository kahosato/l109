In the recent decade, smartphones with a high precision GPS has become the norm. This enabled one to obtain a much more precise locational information than other source, such as the location of the cellular tower that the phone pinged. This has been a valuable source of insight to the human mobility~\citep{rhee2011levy, vazquez2013using, jiang2009characterizing}.
Now, one can know not only where somebody was but also what they were doing, thanks to social networks; various services such as Twitter, Facebook and Instagram allow users to attach the location to their content which often describes the activity that they were engaged in at that location. 

In particular, the data from \emph{Foursquare} has been studied extensively~\citep{scellato2011socio, noulas2011empirical, noulas2013exploiting}. Foursquare is a social network service where a user shares their location by making a \emph{check-in} at a venue. It has been a popular and active platform with 50 million monthly active users and 60 million registered users~\citep{50million}.

The Foursquare dataset that I utilise contains two sets of information. One describes the venues which has at least one check-in from a user and the other describes \emph{transitions} between two check-ins which were made consecutively by the same user. The data was collected in New York City over a period of 12 months from December 2010 to November 2011, and contains 86092 venues and 1678326 transitions. 


The other dataset I analyse is the Airbnb dataset, containing information on 40227 listings in New York. This was compiled from publicly available information on Airbnb website to build \emph{Inside Airbnb}~\citep{insideairbnb}, a tool to explore how Airbnb is used in various cities, including New York City.
The motivation of creating such a tool was to demonstrate how Airbnb could be harmful to the local community by lowering the housing availability and increasing the housing price~\citep{insideairbnb}.
The dataset has been used to expose various other social issues as well, such as gentrification~\citep{gentrification} and digital discrimination~\citep{edelman2014digital}.

In this report, I would like to investigate the notion of \emph{neighbourhood} using these two datasets. The notion of neighbourhood can often be seen in a large city and is usually somewhat fuzzy. It is often the case that there is no well-defined geographical border between them. People do, however, seem to have a general consensus on to which neighbourhood  a given venue belongs to. I argue that the notion of neighbourhood can be defined as a set of venues which are strongly associated with each other. For example, one may consider two ``similar'' shops to be in the same neighbourhood when they are both well-supported by the same socio-economic group which is representative of the neighbourhood. I extract the implicit definition of neighbourhood that the Foursquare dataset suggest by modelling the association between venues using the transitions described in the Foursquare dataset. The Airbnb dataset on the other hand provides its definition of neighbourhood explicitly. Each listing are categorised into a neighbourhood, such as ``Hell's Kitchen'' or ``Upper East Side''. I would like to measure to what extent the notion of neighbourhood extracted from the Foursquare dataset corresponds to the one defined on the Airbnb dataset.

I would also like to investigate how one could estimate the popularity of an area from Airbnb dataset. An indicator for the popularity of an area may be useful for commercial use and academic investigation. For instance, when opening a business one may want to estimate the change in the popularity of an area over time and try to predict an area which is up-and-coming. The total number of check-in has been used as an indicator of the popularity of a venue~\citep{noulas2011empirical}. One may consider an area to be a large Foursquare venue and naturally quantify its popularity as a sum of the number of check-in for Foursquare venues which belong to it. On the contrary, there is no agreed way to quantify the popularity of an area using the information about Airbnb listings. In this report, I compute various statistics on the listings in the neighbourhood extracted from the Foursquare dataset and see if any of them is as good as the sum of the number of check-ins of the Foursquare venues by measuring the correlation between them.

The report proceeds as follows: Section 2 describes the two dataset I investigate in this report and discuss their limitations. Section 3 describes the graph that I constructed from the Foursquare dataset. Section 4 describes some basic analysis carried out on the graph described in Section 3. Section 5 describes the experiment about the notion of neighbourhood. Section 6 investigates various estimators for the popularity of an area that can be used on Airbnb dataset. Finally, Section 7 discusses the possible extension and future works.