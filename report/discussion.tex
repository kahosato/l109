One factor that was not taken into account is the time elapsed since the first check-in till the second check-in in the transition. \cite{noulas2011empirical} states that two temporally close check-ins of the same user could signal an important correlation between
two locations, but as this temporal distance increases we could express higher confidence that the two check-ins are not strictly consequential. \cite{noulas2011empirical} though also demonstrates that there is a positive correlation between inter-check-in times and inter-check-in distances. This means that two check-ins made with a long interval may still be strongly related if they are located further away from each other. As a future work, one may create a more elaborate model for the strength of association by taking such factors into account, which may lead to a better definition of Foursquare neighbourhoods.

We made an assumption that the total check-ins of the venues in an area is an adequate indicator of its popularity. One may investigate a potential bias of this statistic by considering for instance the demographics of the users of Foursquare.