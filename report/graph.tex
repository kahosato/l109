In this section, I specify a weighted undirected graph $\mathcal{G}$ constructed from Foursquare dataset and justify the design choices.
Node in $\mathcal{G}$ represents a venue and an edge exists between two nodes if and only if the dataset contains a transition from one to the other.
A possible interpretation of the edge is association between two places. The assumption is that, if one visits two places one after the other, there must be some relation between two places.

As stated previously, what I aim to extract from this graph is the notion of neighbourhood which consists of venues that people would ``associate'' with each other. This concept of association should be invariant of the order of the visit. Therefore I disregard the direction of the transition and $\mathcal{G}$ is an undirected graph.

I used the weight to represent the strength of association between two venues. In this report, we assume that one associate two locations more strongly if they are geographically close to each other. Based on this assumption, one may define the weight of an edge to be an inversely proportionate to a positive increasing function of distance. 
Though the context is different, some works attempt to relate spatial distance with non-spacial concepts, such as friendship. Letting $d$ to be the distance between two indivisuals, \cite{backstrom2010find} claims that the probability of them having a social connection is proportionate to $d^{-1}$. Others argue that it should be proportionate to $d^{-2}$~\citep{lambiotte2008geographical}. Here we assume that the strength of association between two places decays in proportionate to the inverse of the distance.  For simplification, I define the distance between two locations to be a distance as the crow flies. One may elaborate this by, for instance, using how long it takes from one location to the other according to Google Maps.
